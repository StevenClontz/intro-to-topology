\documentclass[12pt, std]{article}
\usepackage{amssymb}
\begin{document}

Definitions and theorems on countability.

\

Let $\mathbb{Z}^+$ denote the positive integers.

\

Definition.  A function $f: X \rightarrow Y$ is said to be 1-1 (or one-to-one) if and only if whenever $a$ and $b$ are points of $X$ with $a \ne b$ then $f(a) \ne f(b)$.

Definition.  A function $f: X\rightarrow Y$ is said to be \textit{onto} if and only if for each $y$ there is a point $x$ in $X$ so that $f(x) = y$.

Definition.  The set $M$ is said to be \textit{countable} if and only if there is a 1-1 function from $M$ into the positive integers $\mathbb{Z}^+$.


Observation.  A finite set is countable.

\

Theorem CNT 1. The set $M$ is countable if and only if there is a 1-1 function from a subset of the positive integers $\mathbb{Z}^+$ onto $M$.

\


Theorem CNT 2. If the set $M$ is countable and $K \subset M$ then $K$ is countable.

\

Theorem CNT 3. If each of $A$ and $B$ is a countable set then the Cartesian product $A \times B = \{ (a,b) | \ a \in A, b \in B \}$ is countable.

\

Corollary.  The rational numbers $\mathbb{Q}$ is countable.

\

Theorem CNT 4.  If for each positive integer the set $A_i$ is countable, then the set $M = \cup_{n=1}^{\infty} A_n$ is countable.

\





\end{document}
