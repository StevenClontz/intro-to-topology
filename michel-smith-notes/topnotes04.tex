\documentclass[12pt, std]{article}
\usepackage{amssymb}
\begin{document}

Definition. The point $p$ is the \textit{sequential limit point} of
the sequence $x_1, x_2, x_3, ... $ means that if $U$ is an open set
containing $p$ then there exists an integer $N$ so that if $n>N$
then $x_n \in U$.

\

Exercise 4.1.  Consider the reals $\mathbb{R}$ with the standard topology.  Show
that:

\qquad a.  The number $1$ is the sequential limit of the sequence
$\{ \frac {n-1}{n+1} \}_{n=1}^{\infty}$.

\qquad b.  The sequence $\{ (-1)^n +\frac 1n \}_{n=1}^{\infty}$ does
not have a sequential limit point.

\

Theorem 4.1.  If $X$ is Hausdorff then the sequence
$\{x_i\}_{i=1}^{\infty}$ has at most one sequential limit point.

\

Theorem 4.2.  If $X$ is Hausdorff and the point $p$ is the
sequential limit point of the sequence $\{x_i\}_{i=1}^{\infty}$ and
the set $ \{ x_i | i \in \mathbb{Z}^+\}$ is finite, then there
exists an integer $N$ so that $x_n = p$ for all $n > N$.

\

Theorem 4.3.  If $X$ is first countable, $M \subset X$ and $p$ is a limit point of
$M$ then there is an infinite sequence of distinct points of $M$
$\{x_i\}_{i=1}^{\infty}$ so that $p$ is a sequential limit point of
$\{x_i\}_{i=1}^{\infty}$.

\

Exercise 4.2. Show in a metric space that $x$ is the sequential limit point of the sequence $\{x_i\}_{i=1}^{\infty}$ if and only if for each $\epsilon >0$ there exists an integer $N_\epsilon$ so that if $n > N_\epsilon$ then $d(x, x_n) < \epsilon$.

\

Theorem 4.4.  Suppose that $X$ and $Y$ are Hausdorff spaces and $f:
X \rightarrow Y$ is a continuous function and the point $p\in X$ is
the sequential limit point of the sequence $\{x_i\}_{i=1}^{\infty}$.
Then the point $f(p)$ is the sequential limit point of the sequence
$\{f(x_i)\}_{i=1}^{\infty}$.













\end{document}
