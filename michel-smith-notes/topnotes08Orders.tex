\documentclass[12pt, std]{article}
\usepackage{amssymb}
\begin{document}

\begin{center} \textbf{Ordered Topological Spaces.} \end{center}

Definition. Let $S$ be a set.  The relation $<$ is said to be an
\textit{order relation} on $S$ iff:

\qquad 1.  If $a$ and $b$ are two points of $S$ then either $a<b$ or
$b<a$;

\qquad 2.  If $a<b$ and $b<c$, then $a<c$;

\qquad 3.  If $a<b$ then $b \nless a$;

\qquad 4.  For all $a \in S$, $a \nless a$.

\

Notation:  If $<$ is an order relation on the set $S$ then $a \le b$ means $a<b$ or $a=b$.  The relation $\ge$ is similarly defined.

\

Definition.  Suppose that $S$ is a set with the order relation $<$.  If there is a point $\ell$ so that $x \le \ell$ for all $x \in S$, then $\ell$ is called the last point of $S$.

If there is a point $f$ so that $x \ge f$ for all $x \in S$ then $f$ is called the first point of $S$.

\

Definition.  Suppose that $S$ is a set with an order relation $<$
then the order topology on $S$ is constructed as follows:

For $a, b \in S$, define:
\begin{eqnarray*}
(a,b) & = &  \{x | a < x < b\}.
\end{eqnarray*}
\qquad Let $\mathcal{B}$ be the set to which $B$ belong if and only
if $B= (a,b) $ for some pair of elements $a$ and $b$ in $S$, $B=
\{ x \in S | x < a \}$ for some $a \in S$,  or $B= \{ x \in S | b <
x \}$ for some $b\in S$.

Then $\mathcal{B}$ in a basis for the order topology.

\

Theorem 8.1.  Suppose that $S$ is a set with an order relation $<$
then the order topology on $S$ is Hausdorff.

\

Definition.  If $S$ is a set with an order relation $<$ and $M
\subset S$ then $M$ \textit{has a least element} means that there
exists an element $p \in M$ so that $p \le x$ for all $x\in M$.


Definition.  Suppose that $S$ is a set with an order relation $<$.
Then the order relation $<$ is said to be a \textit{well ordering}
if and only if every subset of $S$ has a least element.

\

Exercise 8.2.  The positive integers is well ordered.

\

Exercise 8.3.  The set of rational numbers with the usual ordering
is not well ordered, but a (necessarily different) ordering can be
defined on the rationals which is a well ordering.

\

Exercise 8.4.  The set $\cup_{m=0}^{\infty} \{ m + \frac {n-1}{n} |
n \in \mathbb{Z}^+ \}$ is well ordered by the usual ordering on the
reals.

\

Theorem 8.5.  Every subset of a well ordered set is well ordered.

\

Definition.  If $S$ is a set with the ordering $<$ and $M \subset S$, then $b$ is a \textit{lower bound} for $M$ means $b \le x$ for all $x \in M$; $b$ is a \textit{greatest lower bound} for $M$ means that $B$ is a lower bound for $M$ and if $b' < b$ then $b'$ is not a lower bound for $M$.  $S$ is said to have the \textit{greatest lower bound property} iff whenever $M \subset S$ and $M$ has a lower bound, then $M$ has a greatest lower bound.

The concepts \textit{upper bound}, \textit{least upper bound} and least \textit{upper bound property} are similarly defined.

\

Theorem 8.6.  Let $S$ be a well ordered set with the order topology.
Then $S$ has the greatest lower bound property.

\

Theorem 8.7.  Let $S$ be a well ordered set with the order topology.
Then $S$ has the least upper bound property.

\

Theorem 8.8.  If $S$ is an ordered set with a first and last element and it has the least upper bound property, then $S$ with the order topology is compact.

\

Theorem 8.9.  Let $S$ be a well ordered set with the order
topology which has a last element.  Then $S$ is compact.

\

Theorem 8.10.  There is no infinite decreasing subset of a well
ordered set.

\

Definitions for Exercise 8.10.

Suppose that $A$ and $B$ are two sets with order relations $<_A$ and
$<_B$ respectively.  Then $A$ and $B$ are said to be \textit{order
isomorphic} with respect to these ordering if and only if there is a
1-1 and onto function $f: A \rightarrow B$ so that $x <_A y$ if and
only if $f(x) <_B f(y)$.  If $M$ is a set with order relation $<$
then $I$ is said to be an \textit{initial segment} of $M$ if and
only if $I \subset M$ and if $x \in I$ then $\{ t \in M | \ t< x \}
\subset I$.

\

Exercise 8.11.  Let $G$ denote the collection to which the subset
$g$ of the reals belongs if and only if $g$ is well ordered with
respect to the usual ordering on the reals.  Define the relation
``$\sim$'' on $G$ by $g_1 \sim g_2$ if and only if $g_1$ and $g_2$
are order isomorphic.  Show that $\sim$ is an equivalence relation
on $G$. Let $\mathcal{G}$ be the collection of equivalence classes
of $\sim$; $\mathcal{G} = \{[g]| \ g \in G \}$.  Define $[g_1]
<_{\mathcal{G}} [g_2]$ if and only if $[g_1] \neq [g_2]$ and $g_1$
is order isomorphic to an initial segment of $g_2$.  Show that:

\qquad 1. $<_{\mathcal{G}}$ is an order relation on $\mathcal{G}$,

\qquad 2. $<_{\mathcal{G}}$ is a well ordering,

\qquad 3. $\mathcal{G}$ is uncountable,

\qquad 4. Every initial segment of $\mathcal{G}$ is countable.

\

Theoretical stuff:

\

Axiom of choice.  Suppose that $\mathcal{G}$ is a collection of
sets; then there exists a function $F: \mathcal{G} \rightarrow \cup
\mathcal{G}$ so that $F(g) \in g$ for every $g \in \mathcal{G}$.

\

Well ordering theorem.  The Axiom of choice implies that every set
can be well ordered.



\end{document}
