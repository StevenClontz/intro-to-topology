\documentclass[12pt, std]{article}
\usepackage{amssymb}
\begin{document}
\begin{center} \textbf{Topology notes.} 

\textbf{Basic Definitions and Properties.} 
\end{center}

Intuitively, a topological space consists of a set of points and a
collection of special sets called open sets that provide information
on how these points are related to each other.  One can think of
these points as  a generalization of geometric points and the open
sets as generalizations of geometric regions like the inside of
spheres or cubes.  Thus for the definition of a topological space we
are required to have a set of points for the underlying space and a
collection of these open sets that define the ``topology'' on this
space of points.

\

As a first example, we now define the standard topology of the reals $\mathbb{R}$.  The set $S$ is said to be a segment if and only if there are two numbers $a<b$ so that $S = \{ x \ | \ a<x<b \}$; this segment $S$ is denoted by $(a,b)$.
The set $U$ is said to be open if and only if for each point $p \in
U$ there is a segment $S$ containing $p$ so that $S \subset U$. Let
$\mathcal{T}$ be the collection of all open subsets of $\mathbb{R}$.

\

Exercise 1.1.  Define what it means for a set not to be open.

\

Exercise 1.2.  Determine which of the following sets are open in
$\mathbb{R}$:

\qquad a.  A finite set.

\qquad b.  The complement of a finite set.

\qquad c.  $\{ t | 2t + 1 > 5 \}$

\qquad d.  $\{ t | 2t + 1 \ge 5 \}$

\qquad e.  The integers

\qquad f.  The complement of the integers

\qquad g.  Rational numbers.

\qquad h. The set $\mathbb{R}$.

\qquad i. The empty set $\emptyset$.


\

Exercise 1.3.  Show that the collection of open sets $\mathcal{T}$
have the following properties:

1.) The sets $\mathbb{R}$ and $\emptyset$ are open.

2.) If $A$ is open and $B$ is open, then $A \cap B$ is open.

3.) If $G$ is a collection of open sets, then the union of the
elements of $G$ is an open set.  This set is denoted by $\cup G$ so
that $\cup G = \{ x \in \mathbb{R} | x \in g\textrm{ for some g }\in
G \}$.

\

Exercise 1.4.  In analysis a function $f$ from the reals into the
reals is said to be continuous at the point $p$ if and only if: for
each positive number $\epsilon$ there exists a positive number
$\delta$ so that if: $$|p-x| < \delta$$ then $$|f(p) - f(x)| <
\epsilon.$$

Show that this is equivalent to the statement that $f$ is continuous
at the point $p$ if and only if for every open set $V$ containing
$f(p)$ there is an open set $U$ containing $p$ so that $f$ maps
every point of $U$ into $V$.

Show that if ``open set'' in the above is replaced with ``segment''
then this produces another equivalent definition of continuity at
the point $p$.

\

Intuitively the boundary of an object is something right at the
``edge'' of the object and so is close to the object.  Our intuition
tells us that the boundary of the segment $S =(a,b)$ is the set of points
$\{a, b \}$.  An interval is a segment with these boundary points
added.  We use the notation $[a,b]$ to denote the interval $[a,b] =
\{x \in \mathbb{R} | a \le x \le b \}$.  Our goal in topology is to
define concepts in terms of points and open sets.

Definition.  If $M$ is a set then the boundary of the set $M$,
denoted by $Bd(M)$, is the set to which $p$ belongs if and only if
every open set containing $p$ contains a point in $M$ and a point
not in $M$.

\

Exercise 1.5.  Calculate the boundary of the sets from Exercise 1.2.

\

Exercise 1.6.  Give an example of a set with the property that
neither it nor its complement is open.

\

Theorem 1.1.  If $U \subset \mathbb{R}$ is open, then $U$ does not
contain any of its boundary points.

\

Theorem 1.2.  If $M \subset \mathbb{R}$ then $\mathbb{R} - Bd(M)$ is
open.

\

Building on our analysis of the topology of the reals we define the
concept of a general topological space.

\

Definition. A \textit{topological space} is a pair $(X,
\mathcal{T})$ such that $X$ is of a set of objects called
\textit{points} and $\mathcal{T}$ is a collection of subsets of $X$
such that the following are satisfied:

\qquad 1.  $\emptyset \in \mathcal{T}$ and $X \in \mathcal{T}$;

\qquad 2.  If $A \in \mathcal{T}$ and $B \in \mathcal{T}$ then $A
\cap B \in \mathcal{T}$;

\qquad 3.  If $ \mathcal{U} \subset \mathcal{T}$ then $ \cup \{u | u
\in \mathcal{U} \} \in \mathcal{T} $.

\noindent The elements of the collection $\mathcal{T}$ are called \textit{open
sets} and the collection $\mathcal{T}$ is called the
\textit{topology of $X$.}

\

Since the topology of a space is dependent on the collection of open
sets, the same underlying set of points may have different
topologies.  For any given set of points $X$ we can consider the
following two topologies.

\

Example 1.1.  Let $X$ be a set of points; let $\mathcal{T} = \{
\emptyset, X \}$.

\

Example 1.2.  Let $X$ be a set of points; let $\mathcal{T} = \{ H |
H \subset X \}$.

\

Example 1.1 is often called the trivial or degenerate topology.
Example 1.2 is called the discrete topology on a set.  The first
example describes the minimal topology based on the minimal
requirements for a collection to be a topology on a space.  The
second is the largest topology possible for a set since it contains
all the subsets of the space.

\

Exercise 1.7.  Let $X$ be a topological space.  Suppose that for each
positive integer $i$,  $U_i$ is an open set.  Show that $\cup_{i=1}^{\infty}
U_i$ is open and that if $N$ is an integer then $\cap_{i=1}^{N} U_i$
is open.

\

Exercise 1.8.  Show in the reals that there is a sequence of open
sets $\{U_i\}_{i=1}^{\infty}$ so that $\cap_{i=1}^{\infty} U_i$ is
not open.

\

Since the Euclidean plane can so easily be represented on the
blackboard we define the standard topology for two dimensional
Euclidean space $\mathbb{E}^2$.  Let $X = \{(x,y) | \ x, \in \mathbb{R} \} $ denote the Euclidean plane.  For each point $P = (a,b)$ and $\epsilon > 0$ let $$\mbox{B}_\epsilon(P) = \{ (x,y) | \ \sqrt{(a-x)^2 + (b-y)^2} < \epsilon \}.$$
Then the set $U$ is said to be open in $X$ if and only if for each point $P \in U$ there is a positive number $\epsilon$
so that $\mbox{B}_\epsilon(P) \subset U$.  Let $\mathcal{T}_\circ$ denote this collection of open sets.

\

Exercise 1.9.  Show that $(\mathbb{E}^2, \mathcal{T}_\circ)$ is a topological space.

\

Exercise 1.10.  Let $X$ be the Euclidian plane $\mathbb{E}^2$.  For each point $P = (a,b)$ and $\epsilon > 0$ let $$\hat{ \mbox{B}}_{\epsilon}(P) = \{ (x,y) | \ |x-a| + |y-b| < \epsilon \}.$$
Then the set $U$ is said to be open in $X$ if and only if for each point $P \in U$ there is a positive number $\epsilon$
so that $\hat {\mbox{B}}_\epsilon(P) \subset U$.  Let $\mathcal{T}_\lozenge$ denote this collection of open sets.

Show that $(\mathbb{E}^2, \mathcal{T}_\lozenge)$ is a topological space.

\

Exercise 1.11.  Show that $\mathcal{T}_\circ = \mathcal{T}_\lozenge$.

\

Now we give some common axioms that tell us something about the
relationship of the points of a space and the open sets.

\


Axiom $\textrm{T}_0$.  If $p$ and $q$ are points of $X$ then there
is an open set that contains one of these points and not the other.

\

Axiom $\textrm{T}_1$. If $p$ and $q$ are points of $X$ then there is
an open set that contains $p$ and not $q$.

\

Axiom $\textrm{T}_2$.  If $p$ and $q$ are points of $X$ then there
exist disjoint open sets $A$ and $B$ containing $p$ and $q$
respectively. A topological space that satisfies Axiom
$\textrm{T}_2$ is called a Hausdorff space.

\

Exercise 1.12.  Determine the implications among these three axioms.
In other words determine if it is true that a topological space that
satisfies Axiom $\textrm{T}_i$ also satisfies Axiom $\textrm{T}_j$.
If an axiom $\textrm{T}_i$ space does not satisfy Axiom
$\textrm{T}_j$ then give an example of a set of points $X$ with a
topology $\mathcal{T}$ so that $(X, \mathcal{T})$ satisfies Axiom
$\textrm{T}_i$ but does not satisfy $\textrm{T}_j$.

\

Exercise 1.13.  Let $X$ be a point set.

\qquad Let $\mathcal{T}_1 = \{ \emptyset, X \}$.

\qquad Let $\mathcal{T}_2 = \{U | U \subset X \}$.

Determine whether or not these spaces are Hausdorff.

\

Unless otherwise stated, from this point on assume that all spaces
are Hausdorff.  Furthermore assume that $X$ is always a topological
space with a topology.

\

Definition.  If $(X, \mathcal{T})$ is a topological space and $M
\subset X$ is a point set then the point $p$ is said to be a
\textit{limit point} of the set $M$ if and only if each open set
containing $p$ contains a point of $M$ distinct from $p$.

\

Theorem 1.3.  If $X$ is a finite Hausdorff space then no point of
$X$ is a limit point of $X$.

\

Exercise 1.14.  Give an example of a finite set and three different topologies for that set.

\

Definition.   Suppose $(X, \mathcal{T})$ is a topological space and
$M \subset X$.  Then the \textit{derived set} of $M$ denoted by $M'$
is the set of limit points of $M$.  The \textit{closure} of $M$
denoted by $\overline{M}$ is the set $M \cup M'$.

\

Theorem 1.4.  If $(X, \mathcal{T})$ is a topological space and $M
\subset X$ then $\overline{M} = \overline{\overline{M}}$.

\

Definition.  The set $M \subset X$ is said to be \textit{closed} if
and only if every limit point of $M$ is in $M$.

\

Theorem 1.5.  Suppose $(X, \mathcal{T})$ is a topological space and
$M \subset X$.  Then $M$ is closed if and only if $X-M$ is open.

\

Theorem 1.6.  Suppose $(X, \mathcal{T})$ is a topological space and
$M \subset X$.  Then $M'$ is closed.

\

Theorem 1.7.  If $p$ is a limit point of the set $M$, then every
open set containing $p$ contains infinitely many points of $M$.

\

Theorem 1.8.  If each of $A$ and $B$ is a closed subset of the space
$X$, then $A \cap B$ and $A \cup B$ are closed.

Question: Does Theorem 1.8 hold if $X$ is not required to be
Hausdorff?  Is every space satisfying the conditions of Theorem 1.8
for arbitrary closed sets a Hausdorff space?

\

The definition of the boundary of a set given above in the case of
the reals generalizes to arbitrary topological spaces.  We state the
definition for an arbitrary space.



Definition.  Suppose $(X, \mathcal{T})$ is a topological space and
$M \subset X$.  Then the point $p$ is called a \textit{boundary
point} of $M$ iff every open set containing $p$ contains a point in
$M$ and a point not in $M$.  We denote the boundary of the set $M$
by $Bd (M)$.

\

Exercise 1.15.

\qquad a.  Find an example of a set that has no boundary point.

\qquad b.  Find an example of a set every point of which is a
boundary point.

[Note that as part of this exercise you need to define a space $X$
and a topology on that space, and then find a set in that space that
has the required property.]

\

Theorem 1.9.  If $X$ is a topological space and $M \subset X$, then
$Bd(M) = \overline{M} \cap \overline{(X-M)}$.

\

Corollary 1.9.  If $X$ is a topological space and $M \subset X$,
then $Bd(M)$ is closed.

\

Definition.  Suppose $(X, \mathcal{T})$ is a topological space and
$M \subset X$.  Then the \textit{interior} of $M$, denoted by
$Int(M)$ is the set to which $x$ belongs if and only if there is an
open set containing $x$ lying in $M$.

\

Exercise 1.16.  In all the above theorems determine whether or not
the theorem holds under the weaker $\textrm{T}_0$ or $\textrm{T}_1$
axioms.

\

Exercise 1.17.  Suppose $X$ is a topological space and $A, B, M$ etc. are subsets of $X$. Prove the following.  [In each case determine the weakest axiom
that is needed to prove the statement.  Caution: at least one of
these is false! Note that I may not warn you in the future. In all
the cases where the statement is false you should provide a counter
example.]

\qquad a.  $Int(M)$ is open.

\qquad b.  $Int(Bd(M)) = \emptyset$.

\qquad c.  $Int(Int(M)) = Int(M)$.

\qquad d.  $Bd(Bd(M)) = Bd(M)$.

\qquad e.  $Int(A \cap B) = Int(A) \cap Int (B)$.

\qquad f.  $Bd(A \cap B) = Bd(A) \cap Bd(B)$.

\qquad g.  $Int(M) \cap Int(X-M) = \emptyset$.

\qquad h.  $(X - Int(M)) - Int(X-M) = Bd(M)$.

\qquad i.  $M$ is open iff $M \cap Bd(M) = \emptyset$.

\qquad j. $\overline{A \cup B} = \overline{A} \cup \overline{B}$.

\qquad k. $\overline{A \cap B} = \overline{A} \cap \overline{B}$.

\qquad l. $U$ is open iff $U = Int(U)$.

\qquad m. $X- Int(A) = \overline{X-A}$.

\qquad n. $X= Int(M) \cup Bd(M) \cup Int (X-M)$.

\qquad o. $A$ is closed iff $Bd(A) \subset A$.


















\end{document}
