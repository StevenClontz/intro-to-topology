\documentclass[12pt, std]{article}
\usepackage{amssymb}
\begin{document}

\begin{center} \textbf{Product Spaces.} \end{center}

Definition.  Suppose that each of $X$ and $Y$ is a topological
space.  Then the topological cross product space denoted by $X
\times Y$ is the space whose points are the point pairs: $X \times Y
= \{ (x,y) | x \in X, y \in Y \}$.  A basis for the topology of $X
\times Y$ is obtained as follows: if $U$ is open in $X$ and $V$ is
open in $Y$ then $U \times V$ is a basis element.

\

Theorem 9.1.  If each of $X$ and $Y$ is Hausdorff, then $X \times
Y$ is Hausdorff.

\

Exercise 9.1.  Give an example of an open set in a product space that is
not a basis element as described above.

\

Exercise 9.2.  Reminder: The space $X$ is said to be homeomorphic (or topologically equivalent) to the space $Y$ if and only if there is a 1-to-1 onto map $\varphi: X \to Y$ that is continuous and whose inverse map is also continuous.
 
a.) Let $Z$ be the product space $X \times Y$ and let $p \in
X$.  Show that the subspace $\{(p, y) | y \in Y \}$ is homeomorphic to
$Y$.

b.) Let $Z = X \times X$.  Show that the subspace $\{ (x,x)| x \in
X \}$ is homeomorphic to $X$.

\

Definition.  If $Z$ is the product space $X \times Y$ then the
function $\pi_1 : Z \rightarrow X$ is defined by $\pi_1(x,y) = x$
and the function $\pi_2 : Z \rightarrow Y$ is defined by $\pi_2(x,y)
= y$; these are called the projection maps.

\

Theorem 9.2.  The projection maps $\pi_1$ and $\pi_2$ have the
following properties.

\qquad a.  They are continuous.

\qquad b.  They map closed sets onto closed sets.

\qquad c.  They map open sets onto open sets.

[Note one of these is deliberately false.]

\

Theorem 9.3.  If $S$ is a topological space and $Z = X \times Y$
then $f: S \rightarrow Z$ is continuous if and only if the functions
$ \pi_1 \circ f$ and $\pi_2 \circ f$ are continuous.

\

Theorem 9.4.  If each of $X$ and $Y$ is a metric space then $X
\times Y$ is a metric space.

\

Theorem 9.5.  If each of $X$ and $Y$ is connected then $X \times Y$
is connected.

\

Theorem 9.6.  If one of $X$ and $Y$ is not connected then $X \times
Y$ is not connected.

\

Theorem 9.7.  If each of $X$ and $Y$ is compact then $X \times Y$
is compact.

\

Theorem 9.8.  If there is a well ordering of each of $X$ and $Y$, then there is a well ordering of $X \times
Y$.

\

Note this last theorem is a trivial consequence of the much more
difficult theorem that says under very weak conditions (the Axiom of
Choice) every set can be well ordered.  I expect a proof that doesn't use the (as yet unproven) fact that every set can be well-ordered.  














\end{document}
