\documentclass[12pt, std]{article}
\usepackage{amssymb}
\begin{document}

\begin{center} \textbf{Topological basis.} \end{center}

\

Definition.  Suppose that $(X, \mathcal{T})$ is a topological space.
Then $\mathcal{B} \subset \mathcal{T}$ is said to be a
\textit{basis} for the topology of $X$ provided that for each point
$p \in X$ and each open set $U \in \mathcal{T}$ containing $p$ there is an element
$B \in \mathcal{B}$ so that $ p \in B \subset U$.

\

For the following assume that $(X, \mathcal{T})$ is a topological
space and $\mathcal{B} \subset \mathcal{T}$ is a basis for the
topology of $X$.

\


Theorem 2.1.  The point $p$ is a limit point of $M$ iff every
element of the basis $\mathcal{B}$ containing $p$ contains a point of $M$ distinct
from $p$.

\

Theorem 2.2.  Suppose that $\mathcal{B}$ is a collection of subsets
of $X$ so that:

\qquad 1.  Every element of $X$ is in some element of $\mathcal{B}$.

\qquad 2.  If $p \in X$ and $A$ and $B$ are elements of
$\mathcal{B}$ both containing $p$, then there is an element of
$\mathcal{B}$ containing $p$ lying in $A \cap B$.

Then $\mathcal{T} = \{ \cup W | W \subset \mathcal{B} \}$ is a
topology for $X$.  Definition: Under this hypothesis, the topology
$\mathcal{T}$ is said to be generated by the basis $\mathcal{B}$.

\

Definition.  Let $\mathbb{R}$ denote the reals.  Let $\mathcal{B} =
\{ (a,b) = \{ x| a < x <b \} | a \in \mathbb{R}, b \in \mathbb{R}
\}$.  Then the topology for $\mathbb{R}$ generated by $\mathbb{B}$
is called the standard topology for the reals $\mathbb{R}$.

\

Exercise 2.1.  Let $ \mathbb{E}^2 = \mathbb{R} \times \mathbb{R}$
denote the Euclidean plane. Let $\mathcal{B} = \{ (a,b) \times (c,
d) | a, b, c, d \in \mathbb{R} \}$. Then $\mathcal{B}$ generates the standard topology of $ \mathbb{E}^2$.

\

Theorem 2.3.  Suppose that $(X, \mathcal{T})$ is a topological
space, $Y \subset X$, and $\mathcal{W} = \{ U \cap Y | U \in
\mathcal{T} \}$.  Then $(Y, \mathcal{W})$ is a topological space.

\

Definition. If $X$ and $Y$ satisfy the hypothesis of Theorem 2.3
then $Y$ is said to be a subspace of $X$ with the subspace topology.

\

Corollary 2.4.  If $Y$ is a subspace of the topological space $X$
and $X$ is Hausdorff, then so is $Y$.

\

Example 2.5.  Let $X = \{ x | x \in \mathbb{R} \}$.  If $a \in
\mathbb{R}$ and $b \in \mathbb{R}$ then let $(a,b) = \{ x | \ a < x
< b \}$  Let $\mathcal{B} = \{ (a,b) | a \in \mathbb{R}, b \in
\mathbb{R} \}$. Then the topology generated by the basis
$\mathcal{B}$ is called the standard topology of the reals.

\qquad a.  Show that the reals with the standard topology is
Hausdorff.

\qquad b.  Show that $p$ is a limit point of the set $M$ iff for
each positive number $\epsilon$ there exists a point $x$ in $M$
distinct from $p$ so that $|p-x| < \epsilon $.

\

Exercise 2.6.  Find an example of a topological space which has a
basis of closed sets and such that every point of the space is a
limit point of the space.

\

Definition.  The function $f: X \rightarrow Y$ from the topological
space $X$ to the topological space $Y$ is said to be
\textit{continuous at the point} $x\in X$ if for each open set $V
\subset Y$ containing $f(x)$, there is an open set $U \subset X$
containing $x$ so that every point of $U$ is mapped into $V$ by $f$.
The function $f$ is said to be \textit{continuous} if it is
continuous at each point in its domain.

\

Theorem 2.7. The function $f: X \rightarrow Y$ is continuous iff
$f^{-1}(V)$ is open whenever $V \subset Y$ is open.

\

Exercise 2.8.  Suppose that in the statement of theorem 2.7 the word
``open'' is replaced with the word ``closed''. Then is the statement
still a theorem?

\

Exercise 2.9a.  The function $f: \mathbb{R} \rightarrow \mathbb{R}$
is continuous at the point $a$ iff it is true that if $\epsilon > 0$
then there exists a number $\delta > 0$ so that if $|x-a| < \delta$
then $|f(x) - f(a)| < \epsilon$.

Exercise 2.9b.  The function $f: \mathbb{R} \rightarrow \mathbb{R}$
is continuous iff it is true that if $\epsilon > 0$ then there
exists a number $\delta > 0$ so that if $|x-a| < \delta$ then $|f(x)
- f(a)| < \epsilon$.

\

Theorem 2.10. The function $f: \mathbb{R} \rightarrow \mathbb{R}$ is
continuous iff $f(\overline{A}) \subset \overline{f(A)}$.

\

Theorem 2.11.  If each of $X$, $Y$, and $Z$ is a topological space
and $f: X \rightarrow Y $ and $f: Y \rightarrow Z$ are continuous
functions, then $g \circ f : X \rightarrow Z$ is continuous.

\

Definition.  Let $X$ and $Y$ be topological spaces so that $f: X
\rightarrow Y$ is a continuous 1-1 onto functions whose inverse is
continuous.  Then $X$ and $Y$ are said to be homeomorphic and $f$ is
called a homeomorphism.

\

Exercise 2.12.  Determine which of the following properties are
preserved under (1) onto continuous functions, (2) 1-1 and onto
continuous functions, (3) homeomorphisms.

\qquad a. the space being Hausdorff;

\qquad b. sets being open;

\qquad c. sets being closed;

\qquad d. the boundary of sets;

\qquad e. the interior of sets;

\qquad f. points being limit points of sets.

\

Definition.  If $x$ is a point of the topological space $X$ then the
collection $\mathcal{B}_x$ is a \textit{local basis} at $x$ means
that if $U$ is an open set containing $x$ then there is a member of
$\mathcal{B}_x$ containing $x$ and lying in $U$.

\

Definition.  The space $X$ is said to be \textit{first countable} if
there is a countable local basis at each of its points.

\

Definition.  The space $X$ is said to be \textit{second countable}
or \textit{completely separable} if it has a countable basis.

\

Definition. The set $M \subset X$ is \textit{dense} in $X$ means
that every non-empty open set contains an element of $M$.  The space $X$ is
said to be \textit{separable} if it contains a countable dense set.

\

Exercise.  The reals with the standard topology is first countable,
completely separable and separable.

\

Theorem 2.13.  Suppose that $X$ is a topological space.

\qquad a.  If $X$ is completely separable then it is separable.

\qquad b.  If $X$ is completely separable then it is first
countable.

\

Theorem 2.14.  If $X$ is first countable at the point $x \in X$ then
there exists a sequence of open sets $ \{U_i \}_{i=1}^{\infty}$ so
that $U_i \supset U_{i+1}$ for all positive integers $i$ and $ \{x\}
= \cap_{i=1}^{\infty} U_i $.

\

Exercise 2.15.  Determine which of the following properties are
preserved under (1) onto continuous functions, (2) 1-1 and onto
continuous functions, (3) homeomorphisms.

\qquad a.  Separability.

\qquad b.  Complete separability.

\qquad c.  First countability.








\end{document}
