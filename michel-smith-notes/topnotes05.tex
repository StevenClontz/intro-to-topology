\documentclass[12pt, std]{article}
\usepackage{amssymb}
\begin{document}

Definition.  Suppose that $X$ is a topological space, $G$ is a
collection of subsets of $X$ and $M$ is point set.  Then the
collection $G$ is said to \textit{cover} $M$ if and only if each
point in $M$ lies in some element of $G$.

\

Definition.  Suppose that $X$ is a topological space.  The subset
$M$ of $X$ is said to be \textit{compact} if and only if whenever
$G$ is a collection of open sets so that $G$ covers $M$ then a
finite subcollection of $G$ covers $M$.

\

Exercise 5.1.  Show that the following subsets of the reals is
compact:

\qquad a.  $M= \{x| x = 0 \ \textrm{or} \ x = \frac 1n, \ n \in \mathbb{Z}^+
\}$;

\qquad b.  $M$ is the unit interval $[0,1]$.  [Hint: you need to use the least upper bound axiom of the reals.]

\

Exercise 5.2.  Show that the following subsets of the reals are not
compact:

\qquad a.  $M= \{x|  x = \frac 1n, \ n \in Z^+ \}$;

\qquad b.  $M = \mathbb{R}$;

\qquad c.  $M$ is the open interval $(0,1)$.

\qquad d.  $M$ is the set of positive integers.

\

Assume for the following theorems that all sets lie in a Hausdorff
space $X$.

\

Theorem 5.1.  If $M$ is compact and $H$ is a closed subset of $M$
then $H$ is compact.

\

Theorem 5.2.  If $M$ is compact then it is closed in $X$.

\

Theorem 5.3.  If $f: X \rightarrow Y$ is continuous and $M \subset
X$ is compact then $f(M)$ is compact.

\

Theorem 5.4.  If $M$ is compact and for each positive integer $i$,
$M_i$ is a non-empty closed subset of $M$ so that $M_{i+1} \subset
M_i$, then $\cap_{i=1}^{\infty} M_i \ne \emptyset$.

\

Theorem 5.5.  Suppose that the space $X$ is compact.  Then it is
regular.

\

Theorem 5.6.  Suppose that the space $X$ is compact.  Then it is
normal.

\

Definition.  The collection $G$ of subsets of $X$ is said to be
\textit{monotonic} if and only if for each pair of sets $H$ and $K$
in $G$, one of them is a subset of the other.

\

Theorem 5.7.  Suppose that $M$ is compact and $G$ is a monotonic
collection of non-empty subsets of $M$.  Then there is a point $p$
so that $p$ is a point or a limit point of every set in $G$.

\

Theorem 5.8.  Suppose that $X$ is a metric space, $M$ is a compact
subset of $X$ and $p \in X$.  Then there exists a closest point in
$M$ to $p$.  [I.e. there is a point $q$ in $M$ so that no other
point of $M$ is closer to $p$ than $q$.]

\

Theorem 5.9.  Suppose that $X$ is a metric space and that $H$ and
$K$ are disjoint closed subsets of $X$ and $H$ is compact.  Then
there exists a positive number $\epsilon$ so that $d(h,k) >
\epsilon$ for all $h \in H$ and $k \in K$.

\

Exercise 5.10.  Show that theorem 5.9 is not true if the condition
of compactness is removed.

\

Theorem 5.11.  If $M$ is compact then every infinite subset of $M$
has a limit point.

\

Corollary to theorem 5.11 and exercise 5.1 b [Bolzano�Weierstrass theorem]:  Every infinite and bounded subset of the reals has a limit point.

\end{document}
